\documentclass{jsarticle}

\title{茨城県の高校のネットワークのフィルタリングを \\ スルーしたことについてのレポート}
\author{杉崎 行優}
\date{}

\begin{document}
\maketitle
\abstract{
以下、茨城県の高校のネットワークを当該ネットワークと表記する。

本記事では当該ネットワークのフィルタリングをスルーする方法と手順を述べる。

杉崎は茨城県の高校に通う学生である。
杉崎は本校のネットワークの構築や修復などを教員から頼まれたことがあり、
当該ネットワークについて、一般に知られていない情報を知っているが、本記事ではそれを用いない。
本記事は当該ネットワークを使用することができるならば誰もが手にし得る情報を元に記す。

また、本記事の内容を不正行為や不純行為や犯罪行為等に応用しないよう強くお願いする。
}

\section{当該ネットワークの仕様}
当該ネットワークは NAT されており、内部のネットワークに外部から直接アクセスすることはできない。
また、当該ネットワークから外部ネットワークにアクセスするためには所定のプロキシを通じて通信を行う必要があるが、
外部ネットワーク向きのポートは TCP の HTTP (80番) と HTTPS (443番) しか開いていない。
よって、主にメール送受信のために使われる SMTP (25番) や POP (110番) や IMAP (143番) 等のポートや
他のコンピュータとのセキュアな通信に使われる SSH (22番) 等のポートも、外部ネットワークとの通信では使用することができない。
さらに、プロキシのコンテンツフィルタリングの機能が激しく、ごく一般的な内容のブログ等へのアクセスがブロックされることもある。

\section{フィルタリングをスルーする方法の候補}
フィルタリングをスルーする方法の候補として、SoftEther 等の VPN を当該ネットワークと外部ネットワークの間に設置することが考えられるが、
VPN を設置するためには VLAN インターフェイスを新たに作成する必要があり、
当該ネットワークのような、一般生徒に管理者の権限が与えられていない環境ではこの方法は使用できない。

よって今回は corkscrew というソフトウェアを使用することを考えた。
corkscrew は任意の通信をプロキシを通じて HTTP として偽装するソフトウェアである。
杉崎の自宅のサーバで SSH サービスを起動し、本校のコンピュータから自宅サーバに corkscrew で偽装した SSH プロトコルで接続すると、
当該ネットワークから自宅のネットワークにアクセスすることができる。
また、この通信は SSH なので暗号化されており、セキュアである。

\section{フィルタリングをスルーする手順}
OpenSSH という SSH の実装の1つには、SSH の通信を SOCKS プロキシとして利用することができるようにするオプションがある。
今回、これを用いることによって当該ネットワークから外部ネットワークに、
自宅サーバを通じて任意のプロトコルを用いてアクセスすることができた。

また、私は本校に設置されているサーバ1台を管理している。
しかし、本校の下校時刻は早く、学校にいる時間だけではサーバを十分に管理することができなかった。
よって、自宅サーバから学校内サーバへの OpenSSH によるリバースプロキシを設置した。
当該ネットワークと自宅のネットワーク間の通信は上記の手順により自由に行えるので、
このリバースプロキシを設置するのは容易だった。
結果、自宅からも学校内のサーバにアクセスすることができるようになり、
サーバの管理に十分な時間を割くことができるようになった。

\end{document}
